% !TeX spellcheck = es_es
\documentclass{article}

\usepackage[ansinew]{inputenc}
\usepackage[spanish]{babel}
\usepackage[T1]{fontenc}
\usepackage{fancyhdr}
\usepackage{graphicx}
\usepackage{afterpage}
\usepackage{listings}
\usepackage{minted}
\usepackage[backend=bibtex,style=alphabetic-verb]{biblatex}
\usepackage[a4paper, total={6in, 8in}]{geometry}

\addbibresource{memoria}

\newcommand\blankpage{%
	\null
	\thispagestyle{empty}%
	\addtocounter{page}{-1}%
	\newpage
}

\pagestyle{fancy}
\fancyhf{}
\rhead{Mart��n Romera Sobrado}
\chead{Sistemas de Bases de Datos}
\lhead{PEC Febrero 2021}

\begin{document}
	
	\begin{titlepage}
		\centering
		\vspace*{1cm}
		
		\Huge
		\textbf{Sistemas de Bases de Datos}\\
		\textbf{2021}
		
		\vspace{0.5cm}
		\LARGE
		Centro Asociado de la UNED en Bizkaia\\
		\vspace{1.5cm}
		
		\textbf{Mart�n Romera Sobrado}\\
		\textbf{Bilbao}
		
		\vfill
		
		\vspace{0.8cm}
		\Large
		\today
	\end{titlepage}
	
	\blankpage
	
	\section{Introducci�n}
	
		El objetivo de la pr�ctica es realizar una serie de tareas sobre una base de datos de libros importada de los archivos \texttt{authors.csv} y \texttt{datasets.csv} utilizando el entorno de Hadoop.\\
		
		De las opciones que se ofrec�an en el enunciado para resolver los ejercicios, mi desarrollo ha sido utilizando la \textbf{maquina virtual de cloudera}.\\
		
		Con la entrega de mi pr�ctica adjunto el notebook de jupyter tal y como se pide en los requerimientos, esta memoria explicando brevemente el desarrollo de la pr�ctica y conclusiones de este, el c�digo fuente de los \textit{scripts} utilizados en el desarrollo (aunque se incluyan dentro del notebook en forma de comentarios), en forma de repositorio de git con los cambios que he realizado a lo largo del desarrollo registrados.
		
	\section{Desarrollo}
		
		\subsection{Preparaci�n del entorno}
		
			La preparaci�n del entorno es un paso muy simple, ya que solo consiste en crear un directorio en el sistema de ficheros de hadoop para guardar todos los archivos relacionados con la base de datos, e inicializar la base de datos haciendo uso del fichero \texttt{db.hql} dentro del directorio \texttt{db/}.
			
		\subsection{Ejercicio 1}
		
			En este ejercicio creamos las dos tablas necesarias para el resto de ejercicios: \texttt{authors} y \texttt{dataset}. Las tablas ser�n externas ya que pretendemos dise�ar en un futuro unas funci�n externa (el mapper del ejercicio 5) que interactue con los datos de esta.
			
		\subsection{Ejercicio 2}
		
			En este ejercicio importamos los datos de \texttt{authors.csv} y \texttt{datasets.csv} en las tablas que acabamos de crear. Para ello primero se meten ambos archivos dentro del sistema de ficheros de hadoop, y haciendo uso del script \texttt{ej2.hql} importamos los datos en ambas tamblas.
			
		\subsection{Ejercicio 3}
		
			Para este ejercicio tenemos que crear una vista combinando datos de ambas tablas. La orden para hacer esto es muty similar a como ser�an usando SQL. Esta se recoge en el script \texttt{ej3.hql} dentro del directorio \texttt{db/}.
			
		\subsection{Ejercicio 4}
		
			Este ejercicio consiste en realizar operaciones dentro de la base de datos utilizando el propio lenguaje de consultas de Hadoop, \textit{HiveQL}. Se divide en tres apartados:
			
			\begin{itemize}
				\item[a)] Encontrar el t�tulo del libro m�s alto en el ranking de bestsellers.
				\item[b)] Encontrar el t�tulo del libro con mejor valoraci�n de Ana Mar�a Spagna.
				\item[c)] Encontrar los 5 autores que han escrito m�s libros.
			\end{itemize}
			
			El mayor problema que puede haber en el desarrollo de estos ejercicios es los empates que puede haber a la hora de resolver la consulta.\\
			
			En el apartado \textit{a} no ocurre esto ya que al ser un ranking, suponemos que todos los libros (en el ranking) tienen un valor �nico.\\
			
			En el apartado \textit{b} para evitar que la consulta de 2 resultados, la he basado en ordenar los libros de la autora seg�n su valoraci�n, y mostrar el que est� el inicio de de la tabla resultante de la ordenaci�n con \texttt{LIMIT 1}.\\
			
			En el apartado \textit{c} de hago de forma similar al apartado \textit{b}, ordeno a los autores seg�n el n�mero de libros que han escrito y muestro el ``Top 5'' con \texttt{LIMIT 5}.
			
		\subsection{Ejercicio 5}
		
			Finalmente, para este ejercicio hay que implementar las funciones \texttt{map} y \texttt{reduce} para resolver el apartado \textit{a} del ejercicico anterior. De este ejercicio tengo m�s que hablar, ya que ha estado dandome errores durante casi un mes y medio, y no era capaz de encontrar el problema. Mi problema estaba en que no estaba considerando que hadoop, guardaba los ficheros \textit{csv} partidos en diferentes racks. Esto daba error en mi implementaci�n del \textit{mapper} porque ten�a un sistema para ``autoencontrar'' (para permitir que el programa funcionara para casos m�s generales) las columnas que necesitaba para sus operaciones de la siguiente manera:
			
			\begin{minted}{python}
if first_line:
	# Buscamos en que columna se encuentran los datos de title y bestseller-rank
	column_count = 0
	for column in line:
		if column == "title":
			title = column_count
		if column == "bestsellers-rank":
			rank = column_count
		column_count += 1
		if title == -1 or rank == -1:
			raise Exception("title or rank column not found")
	first_line = False
			\end{minted}
			
			Este condicional se ejecutaba con la primera linea de cada \textit{csv} que recib�a como entrada, y al no encontrar ninguna columna que se llamara o ``title'' o ``bestsellers-rank'', hac�a saltar el error de la linea \mintinline{python}{raise Exception("title or rank column not found")}.\\
			
			Este error, aunque era un error bastante b�sico, me ha sido dificil encontrarlo, ya que al dirigirme al panel de \textit{Hue}, y observaba el trabajo fallido del mapreduce, si buscaba la salida est�ndar de errores (\texttt{stderr}), esta no se mostraba por hacerle falta la activaci�n del componente log4j (supongo que un error de esa compilaci�n de Hadoop).\\
			
			Al final encuentro el problema al ver que en el trabajo se realizan dos ``map''s, uno exitoso y el otro fallido. De lo que deduzco que Hadoop ha dividido el csv en 2 archivos, con uno de ellos sin la cabecera.
			
			El problema lo resuelvo en el commit \texttt{45be2d} del git local, por si se quiere analizar el cambio.\\
			
			Por lo dem�s he utilizado el patr�n ``Top Ten'' de \autocite{mapreducepatters}, solo que adaptado para que en vez de ser un top 10, sea un top 1.\\
					
	\printbibliography
			
\end{document}